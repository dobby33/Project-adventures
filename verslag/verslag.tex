\documentclass[a4paper, 11pt]{article}
\title{Theoretische Informatica: taak adventures}
\usepackage[utf8]{inputenc}
\usepackage{hyperref}						%snelkoppeling in inhoudstafel
\usepackage{float}
\usepackage{graphicx} 						%foto's
\usepackage[usenames,dvipsnames]{color} 	%kleuren van woorden
\hypersetup{
colorlinks,
citecolor=black,
filecolor=black,
linkcolor=black,
urlcolor=black
}

\usepackage[dutch]{babel}
\usepackage[parfill]{parskip}

\usepackage[plain]{fancyref}

\setlength{\textwidth}{0.75\paperwidth}
\setlength{\oddsidemargin}{0cm}
\setlength{\marginparwidth}{0cm}
\setlength{\marginparsep}{0cm}

%header and footer
\usepackage{fancyhdr}
\pagestyle{fancy}
\lhead{Theoretische Informatica}
%header and footer

\begin{document}
	\begin{titlepage}
		\centering
		\vspace{3cm}
		{\scshape \LARGE Theoretische Informatica: taak adventures \par}
		\vspace{6cm}
		{\Large \itshape Tom Martens\par}
		{\Large \itshape Bas Van Assche\par}
		\vspace{2cm}
		\vspace{5cm}
		\vfill
		{\large 25 november 2016}
		
	\end{titlepage}
	
	\pagebreak
	\tableofcontents
	\pagebreak

	\section{Verslag}
		ghjkl

\end{document}